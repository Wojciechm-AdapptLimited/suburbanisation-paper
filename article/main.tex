\documentclass[final,3p,times,authoryear,12pt]{elsarticle}

\usepackage[T1]{fontenc}    % use 8-bit T1 fonts
\usepackage[polish]{babel}
\usepackage{fontspec}
\usepackage{url}
\usepackage{hyperref}
\usepackage{cleveref}
\usepackage{booktabs}
\usepackage{multirow}
\usepackage{array}
\usepackage{lipsum}

\journal{Studia nad Ludnością i Rodzinami}
\abstracttitle{Streszczenie}
\keywordtitle{Słowa kluczowe}
\bibliographystyle{elsarticle-harv}

\begin{document}

\begin{frontmatter}
	\title{Wpływ procesu suburbanizacji na proces formowania się rodzin na przykładzie Bydgoszczy i gmin ościennych}

	\author[uj]{Wojciech Mączka}
	\affiliation[uj]{organization={Instytut Socjologii, Uniwersytet Jagielloński},
		city={Kraków},
		country={Polska}}

	\begin{abstract}
		Artykuł prezentuje eksploracyjną analizę procesu formowania rodzin w Bydgoszczy i gminach ościennych w kontekście procesu suburbanizacji i eksurbanizacji, "rozlewania się"\ miast (\textit{urban sprawl}) w tym rejonie.
	\end{abstract}

	\begin{keyword}
		suburbanizacja \sep małżeństwa \sep rodziny \sep demografia \sep Bydgoszcz
	\end{keyword}
\end{frontmatter}

\section{Wprowadzenie}

Suburbanizacja, czyli zjawisko polegające na wyludnianiu się centrów miast, przy wzroście obszarów podmiejskich, jest jednym z najważniejszych trendów demograficznych w ostatnich latach. Problem "rozlewania się"\ miast podejmowany był zarówno przez badaczy reprezentujących różne dyscypliny naukowe, jak i przez polityków i praktyków \citep{sleszynski2022suburbanizacja}. Suburbanizacja była także przedmiotem konferencji, w tym konferencji zoraganizowanej w ramach III Kongresu Demograficznego \citep{grochowski2023suburbanizacja}.

Celem niniejszego artykułu jest zbadanie wpływu, jaki proces suburbanizacji ma na formowanie się rodzin, ze szczególnym uwzględnieniem wpływu na dzietność. W analizie skupiono się na zbadaniu zmian w przepływach migracyjnych, zawieraniu małżeństw i liczbie urodzeń. Badanym obszarem jest miasto Bydgoszcz i gminy ościenne w powiecie bydgoskim (Białe Błota, Dąbrowa Chełmińska, Nowa Wieś Wielka, Osielsko, Sicienko, Solec Kujawski).

Źródłem wszystkich wykorzystanych w tym artykule danych są bazy danych Głównego Urzędu Statystycznego — Bank Danych Lokalnych (\url{https://bdl.stat.gov.pl/bdl/start}), Dziedzinowe Bazy Wiedzy (\url{https://dbw.stat.gov.pl/}) oraz Baza Danych Demografia (\url{https://demografia.stat.gov.pl/BazaDemografia/StartIntro.aspx})

\section{Skala zjawiska suburbanizacji w badanym rejonie}

Bydgoszcz w 2024 roku liczyła 324 tys. mieszkańców, co w porównaniu do 386 tys. mieszkańców w 1995 roku oznacza spadek o ok. 16.1\%. W tym samym okresie w każdej gminie ościennej w powiecie bydgoskim liczba mieszkańców rosła, choć w zróżnicowanym tempie. Największy wzrost odnotowano w dwóch gminach — Białych Błotach i Osielsku, z odpowiednio 9,2 tys. i 5,8 tys. osób w 1995 roku do 26,5 tys. i 19 tys. osób w 2024 roku, co stanowi wzrost o odpowiednio 280\% i 327\% \citep{bdl2026}. Według prognoz ludności dla gmin na lata 2023-2040 trendy wzrostowe w tych gminach zostaną zachowane, chociaż wzrost będzie wolniejszy niż w latach 1995-2024 \citep{gus2023prognoza}.

Odnotowany wzrost liczby mieszkańców w gminach ościennych w powiecie bydgoskim był efektem przede wszystkim migracji z Bydgoszczy, a cały rejon bydgoski znajduje się w czołówce miast wojewódzkich pod względem stosunku liczby przemeldowań z miasta do strefy podmiejskiej względem innych źródeł napływu, co przedstawiono na mapie \labelcref{fig:struktura-naplywow}.

\begin{figure}[ht]
	\centering
	\includegraphics[width=0.75\textwidth]{img/struktura-naplywow.png}
	\caption{Liczba i struktura rejestrowanych napływów do stref podmiejskich (dla miast powyżej 35 tys. mieszkańców, w tym wszystkich miast na prawach powiatu) w latach 2004-2020 \\ \\
		Źródło: Ministerstwo Funduszy i Polityki Regionalnej, 2022, Krajowa Polityka Miejska 2030, s. 42 (oprac. P. Śleszyński), \url{https://www.gov.pl/web/fundusze-regiony/polityka-miejska}}
	\label{fig:struktura-naplywow}
\end{figure}

Według danych GUS w Bydgoszczy od 20 lat notuje się ujemne saldo migracji, z minimum odnotowanym w 2021 roku wynoszącym poniżej -1900 osób. W zgoła inny sposób prezentowała się sytuacja w gminach ościennych, gdzie saldo migracji w latach 1995-2024 było w większości dodatnie, ze szczególnie wysokimi wartościami w Białych Błotach i Osielsku. Dla wyżek wymienionych gmin saldo migracji pozostawało dodatnie przez cały badany okres, osiągając najwyższe wartości w latach 2003-2007 i 2020-2022. Na rysunku \labelcref{fig:saldo-migracji} przedstawiono saldo migracji w Bydgoszczy, Białych Błotach i Osielsku w latach 2002-2024. Szczególnie interesujące są tu lata 2007-2008 i 2020-2021, w których nastąpiło odwrócenie trendu wymeldowań z Bydgoszczy, co prawdopodobnie było efektem globalnego kryzysu finansowego, który rozpoczął się w Stanach Zjednoczonych w 2007 roku \citep{radomska2013globalny-kryzys-finansowy}, a także pandemii COVID-19 w 2020 roku \citep{kazak2023housing-market-covid-19}.

\begin{figure}[ht]
	\centering
	\includegraphics[width=0.75\textwidth]{img/saldo_migracji.png}
	\caption{Saldo migracji w Bydgoszczy, Białych Błotach i Osielsku w latach 2002-2024. Brak danych dla roku 2015 zaznaczono linią przerywaną. Pionowe linie oznaczają rok 2007 i 2021, w których nastąpiło odwrócenie trendu wymeldowań z Bydgoszczy. \\ \\ Źródło: Opracowanie własne na podstawie danych z Banku Danych Lokalnych GUS \citep{bdl2026}}
	\label{fig:saldo-migracji}
\end{figure}

Tak wysoka intensywność zachodzenia procesu suburbanizacji w badanym rejonie pociąga za sobą negatywne konsekwencje. Jednym z wyzwań, z którymi mierzy się rejon Bydgoszczy, są wysokie koszty związane z infrastrukturą komunikacyjną \citep{piatkowski2019suburbanizacja}. Inne wyzwania związane są z dzietnością. Jak wynika z danych GUS \citep{dbw2026}, a także z reportu o stanie gminy opublikowanego przez urząd gminy Białe Błota \citep{bialeblota2023raport}, liczba dzieci objętych wychowaniem przedszkolnym wzrosła 8-krotnie na przestrzeni ostatnich dekad. W celu zrozumienia podobieństw i różnic w formowaniu się rodzin w badanym rejonie przeprowadzono eksploracyjną analizę danych Głównego Urzędu Statystycznego, której wyniki przedstawiono w następnej sekcji.

\section{Podobieństwa i różnice}

W celu scharakteryzowania wpływu, jaki proces suburbanizacji ma na proces formowania się rodzin, warto najpierw zwrócić uwagę czy na wszystkich badanych obszarach zostały zachowane podobne trendy w zawieraniu małżeństw. W tabeli \labelcref{tab:malzenstwa} przedstawiono liczbę małżeństw zawartych na 1000 ludności w latach 2021-2024 w Bydgoszczy i gminach ościennych w powiecie bydgoskim. W całym badanym okresie różnice w liczbie małżeństw zawartych na 1000 ludności były nieznaczne, choć występowały wahania w zależności od gminy. Warto zaznaczyć, że miasto-rdzeń nie różniło się tu znacząco od gmin ościennych.

\begin{table}[!h]
	\centering
	\resizebox{\ifdim\width>\linewidth\linewidth\else\width\fi}{!}{
		\begin{tabular}{p{0.3\linewidth}>{\raggedleft\arraybackslash}p{0.1\linewidth}>{\raggedleft\arraybackslash}p{0.1\linewidth}>{\raggedleft\arraybackslash}p{0.1\linewidth}>{\raggedleft\arraybackslash}p{0.1\linewidth}}
			\toprule
			\multirow{2}{*}{Gmina} & \multicolumn{4}{c}{Małżeństwa zawarte na 1000 ludności}                      \\
			\cmidrule(l{3pt}r{3pt}){2-5}
			                       & 2021                                                    & 2022 & 2023 & 2024 \\
			\midrule
			Białe Błota            & 4.0                                                     & 4.1  & 3.6  & 3.9  \\
			Dąbrowa Chełmińska     & 4.6                                                     & 4.1  & 3.6  & 2.5  \\
			Nowa Wieś Wielka       & 3.6                                                     & 4.3  & 2.7  & 3.3  \\
			Osielsko               & 3.2                                                     & 4.0  & 3.1  & 4.1  \\
			Sicienko               & 4.9                                                     & 5.6  & 4.1  & 3.3  \\
			Solec Kujawski         & 4.6                                                     & 4.1  & 4.2  & 3.3  \\
			\addlinespace
			Bydgoszcz              & 4.4                                                     & 4.6  & 4.3  & 4.0  \\
			\bottomrule
		\end{tabular}}
	\caption{Małżeństwa zawarte na 1000 ludności w Bydgoszczy i gminach ościennych w latach 2021-2024 \\ \\ Źródło: Bank Danych Lokalnych GUS \citep{bdl2026}}
	\label{tab:malzenstwa}
\end{table}

Inaczej sprawa ma się z dzietnością oraz przyrostem naturalnym. Tutaj różnice między Bydgoszczy a gminami ościennymi są znaczące, choć w ostatnich 5 latach zarówno w powiecie m. Bydgoszcz, jak i w powiecie bydgoskim zanotowano tendencję spadkową. Na rysunku \labelcref{fig:dzietnosc} przedstawiono dzietność w latach 2002-2024 w powiecie m. Bydgoszcz i powiecie bydgoskim.

\begin{figure}[!ht]
	\centering
	\includegraphics[width=0.75\textwidth]{img/dzietnosc.png}
	\caption{Poziom współczynnika dzietności w latach 2002-2024 w powiecie m. Bydgoszcz i powiecie bydgoskim \\ \\ Źródło: Opracowanie własne na podstawie danych z Dziedzinowych Baz Wiedzy GUS \citep{dbw2026}}
	\label{fig:dzietnosc}
\end{figure}

Podobnie sytuacja ma się z liczbą urodzeń na 1000 ludności. Jak wynika z danych GUS, gminy będące suburbiami Bydgoszczy przez większość badanego okresu znacząco przewyższały liczbą urodzeń na 1000 mieszkańców miasto-rdzeń, lecz w ciągu ostatnich 5 lat różnice te się zrównały \citep{bdl2026}.

Zauważalną i stabilną przewagę powiatu bydgoskiego nad Bydgoszczą zauważyć można jednak w przypadku udziału urodzeń wysokiej kolejności (3 i kolejne) w ogólnej liczbie urodzeń. Tutaj udział urodzeń prowadzący do powstania lub utrwalenia wielodzietności był i pozostawał znacząco wyższy przez cały badany okres. Na rysunku \labelcref{fig:kolejnosc_urodzen} przedstawiono udział urodzeń wysokiej kolejności w ogólnej liczbie urodzeń w latach 2002-2024 w powiecie m. Bydgoszcz i powiecie bydgoskim.

\begin{figure}[!ht]
	\centering
	\includegraphics[width=0.8\textwidth]{img/kolejnosc_urodzen.png}
	\caption{Udział urodzeń wysokiej kolejności w ogólnej liczbie urodzeń w latach 2002-2024 w powiecie m. Bydgoszcz i powiecie bydgoskim. Brak danych dla roku 2018 wynika z tego, iż w tym roku jednorazowo karta statystyczna urodzenia nie obejmowała informacji o kolejności urodzenia. \\ \\ Źródło: Opracowanie własne na podstawie danych z Dziedzinowych Baz Wiedzy GUS \citep{dbw2026}}
	\label{fig:kolejnosc_urodzen}
\end{figure}

Przewaga suburbiów Bydgoszczy związanych z dzietnością widoczna jest także w przypadku innych wskaźników demograficznych, takich jak przyrost naturalny. Według danych GUS w latach 2002-2024 przyrost naturalny na 1000 mieszkańców w Bydgoszczy był znacznie niższy niż w gminach ościennych, a w 2023 roku wyniósł on -5,9. Dla kontrastu, w gminach takich jak Białe Błota i Osielsko przyrost naturalny na 1000 mieszkańców wyniósł odpowiednio -0,76 i 0,37. Na tle innych gmin ościennych Bydgoszczy wyjątkowy był tylko Solec Kujawski, w którym przyrost naturalny na 1000 mieszkańców w 2023 roku był niższy niż w Bydgoszczy \citep{bdl2026}. W tabeli \labelcref{tab:przyrost_naturalny} przedstawiono przyrost naturalny na 1000 ludności w Bydgoszczy i gminach ościennych w latach 2021-2024.

\begin{table}[!ht]
	\centering
	\resizebox{\ifdim\width>\linewidth\linewidth\else\width\fi}{!}{
		\begin{tabular}{p{0.3\linewidth}>{\raggedleft\arraybackslash}p{0.1\linewidth}>{\raggedleft\arraybackslash}p{0.1\linewidth}>{\raggedleft\arraybackslash}p{0.1\linewidth}>{\raggedleft\arraybackslash}p{0.1\linewidth}}
			\toprule
			\multirow{2}{*}{Gmina} & \multicolumn{4}{c}{Przyrost naturalny na 1000 ludności}                         \\
			\cmidrule(l{3pt}r{3pt}){2-5}
			                       & 2021                                                    & 2022  & 2023  & 2024  \\
			\midrule
			Białe Błota            & 0.85                                                    & 1.02  & 0.66  & -0.76 \\
			Dąbrowa Chełmińska     & -6.51                                                   & 0.00  & -0.12 & -3.36 \\
			Nowa Wieś Wielka       & -3.15                                                   & 0.78  & -0.29 & -2.12 \\
			Osielsko               & 2.02                                                    & 3.94  & 1.60  & 0.37  \\
			Sicienko               & -0.48                                                   & -1.78 & 1.01  & -1.45 \\
			Solec Kujawski         & -4.46                                                   & -5.27 & -3.48 & -5.89 \\
			\addlinespace
			Bydgoszcz              & -6.90                                                   & -5.84 & -5.94 & -5.67 \\
			\bottomrule
		\end{tabular}}
	\caption{Przyrost naturalny na 1000 ludności w Bydgoszczy i gminach ościennych w latach 2021-2024 \\ \\ Źródło: Bank Danych Lokalnych GUS \citep{bdl2026}}
	\label{tab:przyrost_naturalny}
\end{table}

Podsumowując, choć liczba zawieranych małżeństw w Bydgoszczy była porównywalna z gminami ościennymi, to te wykazały się znacząco wyższą dzietnością i przyrostem naturalnym, właściwie przez zdecydowaną większość badanego okresu. Największą przewagę zauważyć było można w przypadku gmin, które w okresie 2002-2024 wzrosły najbardziej, tj. w Białych Błotach i Osielsku. Dla całego powiatu bydgoskiego dodatkowo zauważalna była przewaga urodzeń wysokiej kolejności w ogólnej liczbie urodzeń, co może być sygnałem świadczącym o tym, że uzyskanie lepszych warunków mieszkaniowych przez pary warunkowało zdecydowanie się na wielodzietność \citep{grochowski2023suburbanizacja}.

\section{Dyskusja}

Z analizy przeprowadzonej w dwóch poprzednich sekcjach wyciągnąć można dwa podstawowe wnioski — rejon Bydgoszczy jest rejonem, w którym proces suburbanizacji zachodzi i zachodził z dużą intensywnością, a także że suburbia Bydgoszczy charakteryzują się znacząco wyższą dzietnością i przyrostem naturalnym niż miasto-rdzeń. Przyczyn takiego stanu rzeczy warto szukać w czynnikach społecznych, w tym tych wynikających z potrzeby uzyskania lepszych warunków mieszkaniowych i dążenia do poprawy jakości życia. Wśród przewag, którymi charakteryzują się suburbia, wyróżnić można, m.in.:

\begin{itemize}
	\item lepsze warunki mieszkaniowe, w tym posiadanie własnego domu jednorodzinnego,
	\item pragnienie kontaktu z naturą,
	\item mieszkanie w miejscu pozbawionym miejskich zanieczyszczeń i tłoku,
	\item atrakcyjność przestrzeni podmiejskiej \citep{degorski2023suburbanizacja}.
\end{itemize}

W kontekście formowania się rodzin i decyzji prokreacyjnych atrakcyjność suburbiów przejawia się przede wszystkim w możliwości uzyskania większej przestrzeni mieszkalnej, która jest kluczowym czynnikiem umożliwiającym posiadanie większej liczby dzieci. Dom jednorodzinny w strefie podmiejskiej oferuje nie tylko więcej metrażu, ale także dostęp do prywatnej przestrzeni zewnętrznej, ogrodu czy podwórka, co jest szczególnie istotne dla rodzin z dziećmi. Jak wynika z przedstawionych wcześniej danych, wyższy udział urodzeń wysokiej kolejności w gminach ościennych Bydgoszczy sugeruje, że poprawa warunków mieszkaniowych poprzez migrację do suburbiów często koreluje z decyzją o powiększeniu rodziny. Przestrzeń podmiejska, charakteryzująca się niższą gęstością zaludnienia i większym dostępem do terenów zielonych, stwarza również bardziej sprzyjające środowisko dla wychowania dzieci, co może wpływać na skłonność par do planowania większej liczby potomstwa.

Warto przy tym zauważyć, że wyższa dzietność w suburbiach może być zarówno efektem migracji rodzin już z dziećmi, jak i decyzji prokreacyjnych podejmowanych przez pary, które osiedliły się w strefie podmiejskiej wcześniej, przed założeniem rodziny lub na początkowym etapie jej formowania. Trudno jednak jednoznacznie, na podstawie jedynie tylko danych statystycznych, odpowiedzieć na pytania dotyczące przyczynowości, tzn. w jakim stopniu zamierzenia prokreacyjne podejmowane przez pary są przyczyną migracji (migracja jest czynnikiem umożliwiającym realizację planów w kontekście posiadania potomstwa), czy też odwrotnie, migracja jako efekt poszerzenia rodziny (migracja jako czynnik poprawiający jakość życia po realizacji zamierzeń prokreacyjnych, dzięki lepszym warunkom mieszkaniowym i środowiskowym) \citep{szukalski2023suburbanizacja}.

\section{Podsumowanie}

Przedstawione w niniejszym artykule dane statystyczne wskazują na wysoką intensywność suburbanizacji w rejonie Bydgoszczy, a także na znacząco wyższą dzietność i przyrost naturalny w gminach ościennych względem miasta. Analiza przepływów migracyjnych w latach 1995-2024 ujawniła systematyczny odpływ ludności z Bydgoszczy, przy jednoczesnym dynamicznym wzroście liczby mieszkańców w gminach ościennych, szczególnie w Białych Błotach i Osielsku.

W zakresie formowania się rodzin zaobserwowano, że wskaźniki zawierania małżeństw w Bydgoszczy i gminach ościennych były porównywalne, co sugeruje, że proces suburbanizacji nie wpływa w istotny sposób na decyzje dotyczące zawierania związków małżeńskich. Istotne różnice ujawniły się natomiast w sferze dzietności — gminy ościenne charakteryzowały się wyższym współczynnikiem dzietności, wyższą liczbą urodzeń na 1000 mieszkańców oraz znacząco wyższym udziałem urodzeń wysokiej kolejności w ogólnej liczbie urodzeń, a szczególnie wyraźna przewaga suburbiów widoczna była w przypadku przyrostu naturalnego.

Ustalenia te wskazują na złożony charakter związku między suburbanizacją a procesami demograficznymi. Zrozumienie mechanizmów stojących za związkami między suburbanizacją a formowaniem się rodzin wymaga dalszych badań, w tym analiz jakościowych pozwalających na lepsze uchwycenie przyczynowości decyzji migracyjnych i prokreacyjnych.

\section*{Notka autorska}

Niniejszy artykuł został napisany z wykorzystaniem edytora tekstu Cursor (\url{https://cursor.com/}), a dostępna w nim funkcjonalność AI została wykorzystana do pracy edytorskiej nad tekstem, oraz, w ograniczonym stopniu, do sugestii dotyczących wprowadzania zmian w tekście.



\bibliography{references}

\end{document}